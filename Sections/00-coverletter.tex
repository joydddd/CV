I am a Ph.D. candidate at the University of Michigan, advised by Prof. Satish Narayanasamy. My research focuses on confidential computing and trusted hardware, with a strong background in GPU architecture and ML systems. \textbf{I'm seeking a research internship position for the spring or summer of 2025. }\par
My research seeks to advance confidential computing solutions for enabling privacy-preserving data analytics solutions ranging from population-scale genomic analysis to generative AI. My approach involves developing trustworthy hardware to efficiently guarantee privacy across system components, removing the operating system and system administrators from the trust base. \par

% \vspace{3ex}

I have made four key contributions. First, I invented the \textbf{Toleo}. Today, trusted processors (Intel SGX) support only a few hundred MBs of secure memory space. Toleo is an innovative solution that expands trust to intelligent memory and scales secure memory space to tens of TeraBytes, which is a million times larger than what is feasible today. This work is accepted to ASPLOS'24. \par


Secondly, during my internship at Meta PyTorch group, I built FlexDecoding, the inference backend for \textbf{FlexAttention}. FlexDecoding addresses the lack of flexibility in supporting new attention variants in today's LLM infrastructure and combines the flexibility of the PyTorch compiler with the performance of expert-tuned decoding kernels. FlexAttention (initially launched with only the training backend in Aug 2024) received 170k views on social media X. FlexDecoding is set to launch in Oct 2024. \par


Thirdly, I contributed to the development of \textbf{SECRET-GWAS}, a privacy-preserving genome-wide association study platform built on Microsoft Azure's confidential computing platform.  SECRET-GWAS scales to over 1000 cores and we demonstrated for the first time that regression analysis on large genomic datasets from multiple institutions can be performed in a few seconds, without exposing data to cloud service provider (under review for Nature Computational Science).\par

Lastly, in collaboration with AMD, I developed new techniques to accelerate long-read genome sequencing (\textbf{mm2-gb}, published in ACM BCB'24). mm2-gb advances computing for genome mapping and alignment, removing computational bottlenecks in the sequencing pipeline to keep up with the ultra-long read trend. Our artifact has gained significant community attention and will soon be released as part of AMD Research Open-source Project. \par

Additionally, I am contributing to \textbf{Timelocked Storage}, a project aimed at minimizing the Trusted Code Base (TCB) for ransomeware defense and excluding human administrators, operating systems, even the main processor from the TCB. \par

% \vspace{3ex}
Looking ahead, I plan to further advance confidential computing to address privacy and safety concerns of generative AI.  The release of NVidia's Hopper confidential computing feature brings GPUs into the trusted hardware family, and offers an exciting opportunity to impact how ML models are trained and used. Current GPU TEE solutions heavily relies on software drivers and limits key optimizations, such as DMA for CPU/GPU communication. I aim to explore solutions that integrate GPU and CPU into a unified TEE, leveragin finer granularity on data movement based by low-level hardware primitives (such as through CXL-IDE feature (Integrity Data Encryption)) to construct a unified trusted CPU-GPU memory. I'm also open to exploring other challenges related to integrating GPUs into the TEE system. \\

% \vspace{3ex}

Please see the next page for my CV and detailed background. \\
\vspace{3ex}
\paragraph*{}
\:\: Sincerely, 
\paragraph*{}
\:\: Joy Dong \par
Ph.D. candidate\par
Computer Science \& Engineering, University of Michigan
