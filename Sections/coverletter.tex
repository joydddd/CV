I'm a \textbf{Ph.D. candidate at the University of Michigan} (advised by Prof. Satish Narayanasamy), working on confidential computing and trusted hardware, with a strong background in GPU architecture and ML systems. I'm looking for a research or engineering internship position during the summer of 2024. \par
My research seeks to advance confidential computing solutions for enabling privacy-preserving data analytics solutions ranging from population-scale genomic analysis to generative AI. My approach is to develop trustworthy hardware, and use it to efficiently guarantee privacy from the rest of system components, including the operating system and system administrators. \par

\vspace{3ex}

I have made four specific contributions. First, I invented the \textbf{Toleo}. Today, trusted processors (Intel SGX) support only a few hundred MBs of secure memory space. Toleo is an innovative solution that expands trust to intelligent memory and scales secure memory space to tens of TeraBytes, which is a million times larger than what is feasible today (ASPLOS'24).\par


Secondly, in collaboration with Meta, I built FlexDecoding, the inference backend for \textbf{FlexAttention}. FlexDecoding addresses the lack of flexibility in supporting new attention variants in today's LLM infrastructure and combines the flexibility of the PyTorch compiler with the performance of expert-tuned decoding kernels. FlexAttention (with only training backend) was launched in August 2024 with 170k views on X. FlexDecoding is set to launch in October 2024.\par


Thirdly, I have helped build \textbf{SECRET-GWAS}, a privacy-preserving genome-wide association study platform on Microsoft Azure's confidential computing platform.  SECRET-GWAS scales to over 1000 cores. We showed for the first time that we can perform regression analysis on large genomic datasets from multiple institutions in less than a few seconds, without revealing data to even the cloud service provider (under review for Nature Computational Science).\par

In addition to these thesis work, in collaboration with AMD, I have also developed new techniques to accelerate long-read genome sequencing (mm2-gb). It will soon be released as part of AMD Research Open-source Project and is accepted to ACM BCB'24. \par


\vspace{3ex}
Going forward, I plan to advance confidential computing to address privacy and safety concerns of generative AI. I'm also looking into providing safe storage against ransomware attacks through trusted hardware. \par
Recent release of NVidia's Hopper confidential computing feature brings GPUs into the trusted hardware family, and provides an opportunity to impact how ML models are trained and used. Current solution for GPU TEE heavily relies on software drivers that disallow important optimizations for CPU/GPU communication. I will seek solutions that integrate GPU and CPU into a unified TEE through low-level hardware primitives (such as through CXL-IDE feature (Integrity Data Encryption)) to allow finer granularity on data movement in order to construct a unified trusted CPU-GPU memory. I'm also open to explore other problems in integrating GPUs into the TEE system. \\

\vspace{3ex}

Please see next page for my CV and detailed background. \\
\paragraph*{}
\begin{flushright} -- Joy\end{flushright}
